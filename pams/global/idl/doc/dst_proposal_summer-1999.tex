\documentstyle[12pt]{article}
\setlength{\topmargin}{0.0pt}
\setlength{\headheight}{0.0pt}
\setlength{\headsep}{0.0pt}
\setlength{\footheight}{0.0pt}
\setlength{\textheight}{8.7truein}
\setlength{\oddsidemargin}{17.0pt}
\setlength{\evensidemargin}{17.0pt}
\setlength{\textwidth}{6.0truein}
\renewcommand{\baselinestretch}{1}

\begin{document}
\begin{centering}
{\Large\bf Proposed Event Reconstruction Summary Tables}  \\
\vspace{0.2in}
{\Large\bf for the Summer 1999 STAR Engineering Run}  \\
\vspace{0.2in}
{\LARGE\bf STAR DST}  \\
\vspace{0.2in}
{\large June 1999}  \\
\vspace{0.3in}

\end{centering}

The STAR DST tables used in MDC1 and MDC2 need to be revised for the
Summer 1999 STAR Engineering run and the following initial physics data
run.  The changes mainly involve clean up of unnecessary variables,
addition of some important quantities, consolidation of tables where
this seems more optimal, and separation of tables into detector specific
structures in the case of the software monitor table.  You will notice 
that each idl file now includes an expanded set of comments at the
beginning which should clarify the tables use and provide documentation
with the data.  In this document
each affected table will be listed, preceded by a brief itemized list
of what was changed and why. 
Following this a brief status report of any
remaining DST level tables not explicitly discussed will be given.

\vspace{0.1in}
\leftline{\bf dst\_run\_header:}
\vspace{0.05in}

The BFC production run summary tables were omitted prior to MDC2.  It is
still important to store the BFC production run summary information with
the other DST data.  Therefore we are reinstating the run summary tables.
The changes in this new version compared to that used in MDC1 are:
\begin{itemize}
\item
{\tt event\_type} is increased to 40 characters to permit better descriptions
of simulation runs.
\item
The BFC production run identifier was given a more descriptive name,
{\tt bfc\_run\_id} to emphasize that this is an offline software production
run, not the experiment run number.
\item
The  experiment run number is loaded in a new variable, {\tt exp\_run\_id}.
\item
I received several emails from trigger people indicating that 4 bytes is
sufficient for the trigger mask word so it is okay.
\item
Geary Eppley suggested that if we are storing beam information that the
average beam polarizations in the longitudinal and transverse directions
could also be saved.  The new variables are listed.  For unpolarized 
beams these are set to 0.0. 
\item
Beam-Beam luminosity is now just an average quantity for the run period
corresponding to the events in the BFC run.
\item
The magnetic field (constant) values used by the tracking detectors for
the BFC run are stored in a new variable, {\tt mag\_field}.
\end{itemize}

\begin{verbatim}

/*  dst_run_header.idl           */    
/*  Table: dst_run_header        */  

/*  This table contains minimal information to identify and briefly
 *  characterize each offline reconstruction production run using the
 *  BFC or similar production chain.  The information for one BFC
 *  production run is put into one row of this table.  This table is
 *  intended for both simulation runs and experimental data reconstruction
 *  runs.
 *
 *  A fundamental requirement is that the event type (e.g. A+A collision,
 *  p+p collision, A+A peripheral, TPC event, EMC only event, calibrations,
 *  etc.) and trigger be the same for all events in the production run.
 *  Average beam polarization magnitudes and average beam-beam collision 
 *  luminosity are included for the experimental run period corresponding
 *  to the events in the production run.  Quantitative values for the 
 *  specific run time and bunch crossing must be obtained from the Conditions 
 *  Data Base.  The magnetic field values used by each tracking detector
 *  in the track fitting procedure is stored here also, where this implies
 *  that only one magnet field map is allowed per BFC production run.
 */

 struct dst_run_header  {
  char   event_type[40];  /* Event type: collision, cosmic, lasers...   */
  long   bfc_run_id;      /* Unique BFC production run ID number        */
  long   exp_run_id;      /* Reference to the experiment run ID number  */
  long   trig_mask;       /* Trigger mask for events in BFC prod. run   */
  long   east_a;          /* A of east moving ion                       */
  long   east_z;          /* Z of east moving ion                       */
  long   west_a;          /* A of west moving ion                       */
  long   west_z;          /* Z of west moving ion                       */
  float  east_pol_L;      /* Avg magnitude of east beam Longitudinal Pol*/
  float  east_pol_T;      /* Avg magnitude of east beam Transverse Pol  */
  float  west_pol_L;      /* Avg magnitude of west beam Longitudinal Pol*/
  float  west_pol_T;      /* Avg magnitude of west beam Transverse Pol  */
  float  sqrt_s;          /* CM total energy per NN pair (GeV)          */
  float  luminosity;      /* Avg luminosity during experiment for events*/
                          /* in production run, 1.0/[cm^2 sec]          */
  float  mag_field[4];    /* Magnetic field used by TPC, SVT-SSD,       */
                          /* FTPC-East and FTPC-West, in Tesla          */
 } ;
/*  Last mod. for dst_run_header:   $Date: 1999/09/27 16:54:15 $ */

\end{verbatim}

\vspace{0.1in}
\leftline{\bf dst\_run\_summary:}
\vspace{0.05in}

Please read the comments at the beginning of this idl file which explain
what it is intended for.  The changes are:

\begin{itemize}
\item
The previous version of this table had a variable to store the software
version number used.  This will go in the data base and is omitted here.
\item
The new BFC production run number variable is used, which is a foreign key
to the {\tt dst\_run\_header} table.
\item
The histogram bin limit values have been moved
into the {\tt dst\_summary\_param}
table.
\item 
The names for the mean and rms values for eta and pT have been simplified.
\item 
The multiplicities are stored in arrays for each detector using the detector
numbering scheme in {\tt StDetectorDefinitions}.
\end{itemize}

\begin{verbatim}

/*  dst_run_summary.idl         */  
/*  Table: dst_run_summary      */  

/*  This table contains crude, overall event reconstruction summary
 *  information about a given BFC production run.  The information for
 *  one BFC production run is put into one row of this table.  This table is
 *  intended for both simulation runs and experimental data reconstruction
 *  runs.
 *
 *  Note that the production software version, run control parameters,
 *  calibrations constants, magnetic field map, etc. are to be accessed
 *  through the STAR Data Base, not here.
 *
 *  The mean and standard deviation multiplicities (or energy deposition
 *  in the case of the EMC) for each detector subsystem for all events
 *  successfully processed in the BFC production run are stored in the
 *  variables mean_mult and rms_mult where the array index used
 *  for each detector is determined by the detector ID assignment defined
 *  in StDetectorDefinitions.h .
 */

 struct dst_run_summary  {
  long   bfc_run_id;      /* Unique BFC run ID, F.key to dst_run_header    */
  long   n_events_tot;    /* Total number of events in the BFC prod. run   */
  long   n_events_good;   /* Total number events successfully processed    */
  long   date[2];         /* Start/stop date for processing                */
  long   time[2];         /* Start/stop time of day (sec)                  */
  float  cpu_total;       /* Total cpu sec for production run              */
  float  eta[2];          /* Mean and std.dev. of <eta> for all events     */
  float  pt[2];           /* Mean and std.dev. of <pt> for all events      */
  float  num_vert[2];     /* Mean and std.dev. of total # vertices         */
  float  mean_mult[30];   /* Mean multiplicity (energy) per detector for run */ 
  float  rms_mult[30];    /* RMS multiplicity (energy) per detector for run  */
 } ;
/*  Last mod. for dst_event:   $Date: 1999/09/27 16:54:15 $ */

\end{verbatim}

\vspace{0.1in}
\leftline{\bf dst\_summary\_param:}
\vspace{0.05in}

After a lengthy discussion at BNL we concluded that ROOT generated histograms
for the event summary information could not be easily stored with the DSTs
and we went back to the idea of using modules and tables for this.  Then
the plan to use a variable dimension for the number of bins was opposed 
since this would create incompatibilities among different DST data files.
So the simplest solution, suggested by Kathy, is adopted here.  The number
of bins is increased from 5 to 10, allowing the BFC production manager
to use up to this amount or fewer by simply setting the appropriate bin
ranges in this table in the BFC Maker.  The BFC run number ID was added
and the azimuthal bins can now be variable.

\begin{verbatim}

/*  dst_summary_param.idl         */  
/*  Table: dst_summary_param      */  

/*  This table contains the bin domain parameters used in the event summary tabl
e
 *  for all events processed in the BFC production run.  The information
 *  for one BFC production run is put into one row of this table.  This
 *  table is intended for both simulation runs and experimental data
 *  reconstruction runs.
 *
 *  Up to 10 bins in pt, eta and phi are allowed.  The following define
 *  the domains of the bins; the numerical values must be listed in
 *  monotonically increasing order.  If fewer than 10 bins are needed
 *  the user may simply set the last several parameters to the same value.
 *
 *  Bin domains are defined as follows:
 *
 *  For pt and eta:
 * ----------------
 *
 *          Bin #           Domain
 *            1           <= bin[0]
 *            2            > bin[0] && <= bin[1]
 *           ...             ...
 *            n            > bin[n-2]
 *
 *  For phi (azimuthal angle):
 *  --------------------------
 *
 *          Bin #           Domain
 *            1            > bin[0] && <= bin[1]
 *            2            > bin[1] && <= bin[2]
 *           ...             ...
 *            n            > bin[n-1] && <= bin[0]
 */

 struct dst_summary_param  {
  long   bfc_run_id;      /* Unique BFC run ID, F.key to dst_run_header    */
  float  eta_bins[9];     /* Pseudorapidity bin domain limits              */
  float  pt_bins[9];      /* Transverse momentum bin domain limits (GeV/c) */
  float  phi_bins[10];    /* Azimuthal angle bin domain limits (deg)       */ 
 } ;
/*  Last mod. for dst_event:   $Date: 1999/09/27 16:54:15 $ */

\end{verbatim}

\vspace{0.1in}
\leftline{\bf dst\_event\_header:}
\vspace{0.05in}

\begin{itemize}
\item
The event type name was increased to allow up to 40 characters.
This is retained until we are assured that the events in a BFC run
will always be of the same event/trigger type.  If this turns out to
be always true then this variable can be deleted since it would then
be redundant with that in {\tt dst\_run\_header}.
\item
The experimental run ID name was changed as in the {\tt dst\_run\_header}
table.
\end{itemize}

\begin{verbatim}

/*  dst_event_header.idl        */
/*  Table: dst_event_header     */

/*  This table contains minimal information to uniquely identify an
 *  event.  The information in this table does not depend on results
 *  of any offline analysis and is fixed for all time.  It is filled
 *  with information for one event per table row.  This table is intended
 *  for both simulations and experimental data.
 */

 struct dst_event_header  {
  char   event_type[40];    /* Event type: collision, cosmic, lasers...    */
  long   n_event[2];        /* Unique 64 bit event ID number               */
  long   exp_run_id;        /* Reference to the experiment run ID number  */
  long   time;              /* Unique time stamp for event                 */
  long   trig_mask;         /* Trigger mask                                */
  long   bunch_cross;       /* Beam-Beam bunch crossing number             */
 } ;
/*  Last mod. for dst_event:   $Date: 1999/09/27 16:54:15 $ */

\end{verbatim}

\vspace{0.1in}
\leftline{\bf dst\_event\_summary:}
\vspace{0.05in}

This table has a lot in common with the Physics Working Group StEvent tags
which are also going to be filled during reconstruction.  The following
table was modified to avoid duplication as much as possible and to mainly
emphasize the overall characteristics of the event as determined by the
tracking software.  The intent is that these quantities be used as TagDB
data from the reconstruction phase of the analysis.  I had some discussion
with Mike Lisa and Raimond Snellings about the reaction plane.  I decided
not to include any reaction plane information here but to rely on the
StFlowMaker to take care of this.  Please read the
introductory comments in the idl file.

\begin{itemize}
\item
The {\tt bfc\_run\_id} name is used.
\item
A new global track counter, {\tt glb\_trk\_exotic}, was added to allow for
rare particle searches, such as Jay Marx and David Hardtke propose, or
others that may not be included in the PWG tags.  
\item
The different types of vertices are recorded by the Strangeness tags,
but the total and the totals of each type are put here.
\item 
The ill-defined vertex quality variable was omitted. 
\item
The multiplicity histograms are increased to allow up to 10 bins.
\item
The slope parameters ``Temperature'' are omitted.
\item
The EMC energy bins are increased to 10.
\item
The primary vertex covariance matrix and chi-square were moved to the new
{\tt dst\_monitor\_soft\_global} table.
\end{itemize}

\begin{verbatim}

/*  dst_event_summary.idl       */
/*  Table: dst_event_summary    */

/*  This table contains event reconstruction summary information obtained
 *  during BFC production runs.  It is filled with information for one event
 *  per table row.  This table is intended for both simulations and 
 *  experimental data.  The data contained here are, for the most part,
 *  complementary to that filled by the Physics Working Group Makers into
 *  the PWG Tags in StEvent, also during event reconstruction.
 *
 *  NOTE:  The following should not be construed in any way as representing 
 *         "Physics Quantities"
 *
 *  Other details:
 *        The global track count uses the first instance of the 
 *        dst_track table only, i.e. the one using only the 
 *        hits in the tracking detectors and not the primary vertex.
 */

 struct dst_event_summary  {
  long   n_event[2];        /* Unique 64 bit event ID number, F.key to     */
                            /*    dst_event_header                         */
  long   bfc_run_id;        /* Unique BFC run ID, F.key to dst_run_header  */
  long   glb_trk_tot;       /* Total number of global tracks               */
  long   glb_trk_good;      /* Total number of good global tracks          */
  long   glb_trk_prim;      /* Total number of good, primary global track  */
                            /*    candidates as determined in evr.         */
  long   glb_trk_plus;      /* Total number of good global (+) tracks      */
  long   glb_trk_minus;     /* Total number of good global (-) tracks      */
  long   glb_trk_exotic;    /* Total number of good global exotic trk cand.*/
  long   n_vert_total;      /* Total number of vertices, loose cuts        */
  long   n_vert_type[5];    /* Total number of vertices, loose cuts for    */
                            /*    the five vertex types defined in         */
                            /*    StDetectorDefinitions.h                  */
  long   n_vert_pileup;     /* Number of suspected pileup vertices         */
  long   mult_eta[10];      /* # good glb chrg trk, all pt for eta bins    */
  long   mult_pt[10];       /* # good glb chrg trk, all eta for pt bins    */
  long   mult_phi[10];      /* # good glb chrg trk, all pt,eta for phi bins*/
  float  mean_pt;           /* Mean pt(GeV/c) for all good glb chrg tracks */
  float  mean_pt2;          /* Mean pt^2(GeV/c)^2 for all good glb chrg trk*/
  float  mean_eta;          /* Mean pseudorap. for all good glb chrg trks  */
  float  rms_eta;           /* RMS pseudorap. for all good glb chrg trks   */
  float  energy_emc_eta[10]; /* EMC energy (or raw ADC sum) for eta bins    */
  float  energy_emc_phi[10]; /* EMC energy (or raw ADC sum) for phi bins    */
  float  prim_vrtx[3];      /* Primary vrtx x,y,z-position -SCS(cm)        */
 } ;
/*  Last mod. for dst_event_summary:   $Date: 1999/09/27 16:54:15 $ */

\end{verbatim}

\vspace{0.1in}
\leftline{\bf Software Production Monitor Tables:}
\vspace{0.05in}

The big change here is to separate the former table, {\tt dst\_monitor\_soft}
into different tables for each detector plus one for global reconstruction
quantities.  So far all I have done is move the variables in the previous
table to the new tables.  The TPC and Global tables should be
fairly complete.   The rest need some work from the detector representatives.
The SVT and FTPC tables are perhaps okay now and we can go with this but
Helen and Janet ought to review them.  The EMC, CTB-TOF and RICH tables
are just place-holders and should not go in the production chain until 
something reasonable is put in.

\begin{verbatim}

/*  dst_monitor_soft_tpc.idl           */
/*  Table: dst_monitor_soft_tpc        */

/*  This table contains TPC event reconstruction monitoring quantities
 *  which are used to determine the general quality of the event
 *  reconstruction production chain (BFC).  It is filled with information
 *  for one event per table row.  This table is intended for both
 *  simulations and experimental data.
 */

 struct dst_monitor_soft_tpc  {
  long   n_event[2];         /* Unique 64 bit event ID number;       */
                             /* F. key to dst_event_header           */
  long   bfc_run_id;         /* Unique BFC run ID, F.key to dst_run_header  */
  long   n_clus_tpc_tot;     /* Total number clusters in TPC         */
  long   n_clus_tpc_in[24];  /* Tot. # clus in TPC, inner sectors    */
  long   n_clus_tpc_out[24]; /* Tot. # clus in TPC, outer sectors    */
  long   n_pts_tpc_tot;      /* Tot. # space pts in TPC              */
  long   n_pts_tpc_in[24];   /* Tot. # space pts in TPC, inner sect. */
  long   n_pts_tpc_out[24];  /* Tot. # space pts in TPC, outer sect. */
  long   n_trk_tpc[2];       /* Total # tracks in TPC; tanl<0,=>0    */
  float  chrg_tpc_drift[10]; /* Charge deposited in TPC in along z   */
  float  chrg_tpc_tot;       /* Tot. charge deposition in TPC        */
  float  chrg_tpc_in[24];    /* Tot. charge dep. in TPC, inner sect. */
  float  chrg_tpc_out[24];   /* Tot. charge dep. in TPC, outer sect. */
  float  hit_frac_tpc[2];    /* Frac. hits used in TPC, tanl<0,=>0   */
  float  avg_trkL_tpc[2];    /* Avg. track length (cm) TPC,          */
                             /* -OR- Avg. # pts assigned,  tanl<0,=>0*/
  float  res_pad_tpc[2];     /* Avg. resid, pad direct, TPC          */
                             /* -OR- Avg. chisq(1) of fit,tanl<0,=>0 */
  float  res_drf_tpc[2];     /* Avg. resid, drift dir,  TPC          */
                             /* -OR- Avg. chisq(2) of fit,tanl<0,=>0 */
 } ;
/*  Last mod. for dst_monitor_soft_tpc:   $Date: 1999/09/27 16:54:15 $ */




/*  dst_monitor_soft_svt.idl           */
/*  Table: dst_monitor_soft_svt        */

/*  This table contains SVT-SSD event reconstruction monitoring quantities
 *  which are used to determine the general quality of the event
 *  reconstruction production chain (BFC).  It is filled with information
 *  for one event per table row.  This table is intended for both
 *  simulations and experimental data.
 */

 struct dst_monitor_soft_svt  {
  long   n_event[2];         /* Unique 64 bit event ID number;       */
                             /* F. key to dst_event_header           */
  long   bfc_run_id;         /* Unique BFC run ID, F.key to dst_run_header  */
  long   n_clus_svt;         /* Total number clusters in SVT         */
  long   n_pts_svt;          /* Tot. # space pts in SVT              */
  long   n_trk_svt;          /* Total # tracks in SVT                */
  float  chrg_svt_tot;       /* Tot. charge deposition in SVT        */
  float  hit_frac_svt;       /* Frac. hits used in SVT               */
  float  avg_trkL_svt;       /* Avg. track length (cm) SVT           */
                             /* -OR- Avg. # pts assigned             */ 
  float  res_pad_svt;        /* Avg. residual, pad direction, SVT    */
                             /* -OR- Avg. chisq(1) of fit,           */
  float  res_drf_svt;        /* Avg. resid., drift direction, SVT    */
                             /* -OR- Avg. chisq(2) of fit,           */
 } ;
/*  Last mod. for dst_monitor_soft_svt:   $Date: 1999/09/27 16:54:15 $ */




/*  dst_monitor_soft_ftpc.idl           */
/*  Table: dst_monitor_soft_ftpc        */

/*  This table contains FTPC event reconstruction monitoring quantities
 *  which are used to determine the general quality of the event
 *  reconstruction production chain (BFC).  It is filled with information
 *  for one event per table row.  This table is intended for both
 *  simulations and experimental data.
 */

 struct dst_monitor_soft_ftpc  {
  long   n_event[2];         /* Unique 64 bit event ID number;       */
                             /* F. key to dst_event_header           */
  long   bfc_run_id;         /* Unique BFC run ID, F.key to dst_run_header  */
  long   n_clus_ftpc[2];     /* Tot. # clus in FTPC, east/west       */
  long   n_pts_ftpc[2];      /* Tot. # space pts in FTPC, east/west  */
  long   n_trk_ftpc[2];      /* Total # tracks in FTPC east/west     */
  float  chrg_ftpc_tot[2];   /* Tot. charge dep. in FTPC, east/west  */
  float  hit_frac_ftpc[2];   /* Frac. hits used in FTPC, east/west   */
  float  avg_trkL_ftpc[2];   /* Avg. track length (cm) FTPC,east/west*/
                             /* -OR- Avg. # pts assigned             */ 
  float  res_pad_ftpc[2];    /* Avg. residual, pad direction,FTPC E/W*/
  float  res_drf_ftpc[2];    /* Avg. resid., drift direction,FTPC E/W*/
 } ;
/*  Last mod. for dst_monitor_soft_ftpc:   $Date: 1999/09/27 16:54:15 $ */




/*  dst_monitor_soft_emc.idl           */
/*  Table: dst_monitor_soft_emc        */

/*  This table contains EMC event reconstruction monitoring quantities
 *  which are used to determine the general quality of the event
 *  reconstruction production chain (BFC).  It is filled with information
 *  for one event per table row.  This table is intended for both
 *  simulations and experimental data.
 */

 struct dst_monitor_soft_emc  {
  long   n_event[2];         /* Unique 64 bit event ID number;       */
                             /* F. key to dst_event_header           */
  long   bfc_run_id;         /* Unique BFC run ID, F.key to dst_run_header  */
  float  energy_emc;         /* Total energy (or ADC sum) in EMC     */
 } ;
/*  Last mod. for dst_monitor_soft_emc:   $Date: 1999/09/27 16:54:15 $ */




/*  dst_monitor_soft_ctb-tof.idl           */
/*  Table: dst_monitor_soft_ctb-tof        */

/*  This table contains CTB-TOF event reconstruction monitoring quantities
 *  which are used to determine the general quality of the event
 *  reconstruction production chain (BFC).  It is filled with information
 *  for one event per table row.  This table is intended for both
 *  simulations and experimental data.
 */

 struct dst_monitor_soft_ctb-tof  {
  long   n_event[2];         /* Unique 64 bit event ID number;       */
                             /* F. key to dst_event_header           */
  long   bfc_run_id;         /* Unique BFC run ID, F.key to dst_run_header  */
  long   mult_ctb_tot;       /* Total mult. (or ADC sum) in CTB      */ 
 } ;
/*  Last mod. for dst_monitor_soft_ctb-tof:   $Date: 1999/09/27 16:54:15 $ */




/*  dst_monitor_soft_rich.idl           */
/*  Table: dst_monitor_soft_rich        */

/*  This table contains RICH event reconstruction monitoring quantities
 *  which are used to determine the general quality of the event
 *  reconstruction production chain (BFC).  It is filled with information
 *  for one event per table row.  This table is intended for both
 *  simulations and experimental data.
 */

 struct dst_monitor_soft_rich  {
  long   n_event[2];         /* Unique 64 bit event ID number;       */
                             /* F. key to dst_event_header           */
  long   bfc_run_id;         /* Unique BFC run ID, F.key to dst_run_header  */
  long   mult_rich_tot;      /* Total mult. (or ADC sum) in RICH      */ 
 } ;
/*  Last mod. for dst_monitor_soft_rich:   $Date: 1999/09/27 16:54:15 $ */




/*  dst_monitor_soft_global.idl           */
/*  Table: dst_monitor_soft_global        */

/*  This table contains Global event reconstruction monitoring quantities
 *  which are used to determine the general quality of the event
 *  reconstruction production chain (BFC).  It is filled with information
 *  for one event per table row.  This table is intended for both
 *  simulations and experimental data.
 */

 struct dst_monitor_soft_global  {
  long   n_event[2];         /* Unique 64 bit event ID number;            */
                             /* F. key to dst_event_header                */
  long   bfc_run_id;         /* Unique BFC run ID, F.key to dst_run_header*/
  long   n_trk_match[2];     /* Total # SVT-TPC tracks matched,tanl<0,=>0 */
  long   prim_vrtx_ntrk;     /* # tracks used in primary vertex fit       */
  float  prim_vrtx_cov[6];   /* Primary vrtx covariance matrix            */
  float  prim_vrtx_chisq;    /* Primary vertex chi-square of fit          */
 } ;
/*  Last mod. for dst_monitor_soft_global:   $Date: 1999/09/27 16:54:15 $ */

\end{verbatim}

\vspace{0.1in}
\leftline{\bf dst\_pixel:}
\vspace{0.05in}

This is a new DST table to hold ADC data associated with rare particle
candidates.  A requirement for this table's existence is that it not be allowed
to swell to enormous volume.  It contains two packed data words which
contain the detector ID, sector number, pad row, pad, time bucket and ADC
values.  If hits in other detectors contribute this has to be similarly packed.

With respect to the fractional charge search proposed by Jay Marx and David
Hardtke
the rare particle track count can be saved in
{\tt dst\_event\_summary.glb\_trk\_exotic}.  The  dE/dx values for candidate
tracks can be saved in {\tt dst\_dedx} and can be identified as such using
the {\tt dst\_dedx.iflag} variable.  This will provide a record of the 
rare track candidates via the foreign key, {\tt dst\_dedx.id\_track} which
refers to the particles in {\tt dst\_track}.

\begin{verbatim}

/*  dst_pixel.idl         */
/*  Table: dst_pixel      */

/*  This table contains event reconstruction pixel information.
 *  This might include, for example, the raw ADCs associated with
 *  some rare track candidates from a search algorithm such as that
 *  proposed to look for fractionally charged particles.  It is filled
 *  with information for one event with one pixel per table row.  No
 *  keys to clusters, space points or tracks are included in this table.
 *  Data packing is used for each variable as follows:
 *
 *  secrowpad packs the detector ID number using the definitions in
 *            StDetectorDefinitions (4 bits), the sector number 1 to
 *            24 (5 bits), the pad row number 1 to 45 (6 bits) and the
 *            pad number (maximum of 182) (8 bits).  Similar packing
 *            schemes must be used for other detectors.
 *
 *  timeadc   packs the time bucket number and ADC value.
 */

 struct dst_pixel  {
  long  secrowpad;    /* Packed det ID, sec., row., pad numbers       */
  long  timeadc;      /* Packed time bucket and ADC                   */
 } ;
/*  Last mod. for dst_pixel::   $Date: 1999/09/27 16:54:15 $ */

\end{verbatim}

\vspace{0.1in}
\leftline{\bf dst\_point:}
\vspace{0.05in}

The only change in this table is to reduce the track ID foreign key to
a `short.'  The comments in the idl file emphasize that all space points
are stored and that the track ID refers to the tracks in the first
instance of {\tt dst\_track}, and not any other instances that may be
written out. The problem is that for the track fits including the
primary vertex, some tracks will have space points removed.  We do not
have a mechanism to record this in the DST and I don't know of a good
reason why we should.  Also, Torre advised that as long as we are saving
xdf files the packing algorithm should be maintained.  Any ROOT data
compression is irrelevant for xdf data storage.

\begin{verbatim}

/*  dst_point.idl                 */
/*  Table: dst_point              */

/*  This table contains event reconstruction information about the
 *  space points generated by each tracking detector; TPC, SVT, SSD,
 *  and the FTPC-East, West.  It is filled with information for one event
 *  with one space point per table row.  All space points are loaded
 *  into this table.  Data packing is used for each variable except
 *  id_track which must be fit into a 'short' word length.  Refer to
 *  the dst_point_filler.F module for the data packing algorithm.
 *  The space points can be unpacked using module dst_point_unpack.F
 *  or StEvent.
 *
 *  The foreign key 'id_track' refers to the tracks in the first
 *  instantiation of the dst_track table which involves track fits
 *  to detector space points only, i.e. the primary vertex is NOT
 *  included in the fits.  The relationship between the space points
 *  and the tracks fitted with the primary vertex (second instantiation 
 *  of dst_track) is not saved in the DST.  The relationships between the
 *  space points and any other instantiation of dst_track are not saved
 *  in the DST.
 *
 *  For space points not assigned to tracks in the first instantiation
 *  of the dst_track table, the id_track foreign key is set to 0.
 */

 struct dst_point  {
  short     id_track;      /* F.key into dst_tracks table          */
  long      hw_position;   /* detector, sec, row, pad, time sizes  */
  long      position[2];   /* compressed x,y,z of space-point      */
  long      pos_err[2];    /* compressed x,y,z space-point error   */
  long      charge;        /* total charge and max_adc             */
 } ;
/*  Last mod. for dst_point:   $Date: 1999/09/27 16:54:15 $ */

\end{verbatim}

\vspace{0.1in}
\leftline{\bf dst\_track:}
\vspace{0.05in}

\begin{itemize}
\item
Reduced the primary key word size to `short.'
\item
Added the Geant PID number to allow the Kalman filter/fitter analysis 
a place to store the mass assumption.  The Geant 3.xx numbering system
is preferred, rather than the Particle Data Group's scheme, since we 
need to be able to track ions.  The many particles which are included in the PDG
scheme, but not in the Geant scheme, are not relevant for tracking.
\item
The full covariance matrix is included and the Kalman filter program's
definition adopted.  
\item
The number of degrees of freedom variable was omitted.
\item
The chi-square variable is defined as per d.o.f.
\item
The PID, EMC and SMD foreign keys were omitted.
\item
The track stop vertex ID was omitted.
\end{itemize}

\begin{verbatim}

/*  dst_track.idl              */
/*  Table: dst_track           */

/*  This table contains event reconstruction information about the
 *  charged particle trajectories (tracks) generated by each tracking detector;
 *  TPC, SVT, SSD, and the FTPC-East, West, which have been assigned the
 *  status of global tracks by global tracking analysis.  It is filled with 
 *  information for one event with one track per table row.  Track
 *  parameters are given in the global STAR coordinate system at various
 *  positions depending on the instantiation of this table.
 *
 *  Multiple instances of this table may be saved in the DST output.
 *  The first instance should correspond to tracks that use only the
 *  detector hits with the track parameters given at the first point.
 *  Refer to St_global_Maker for other specific instantiations.
 *  
 *  Description of selected variables:
 *  ----------------------------------
 *       det_id          Indicates which detector(s) contributed space
 *                       points to the track using the detector ID and
 *                       global track definitions in StDetectorDefinitions.h
 *
 *       pid             Geant particle ID number for mass hypothesis
 *                       used in the tracking.  This is included for the
 *                       Kalman filter/fitter method.  The Geant PID code
 *                       is used here, rather than the Particle Data
 *                       Group's code, in order to include deuterons,
 *                       tritons, alphas.
 *
 *       covar[15]       Track fitting covariance matrix.  Kalman filter
 *                       code definition adopted here, where:
 *
 *                       ______|________________
 *                       phi*R |  1  2  3  4  5
 *                        z0   |  2  6  7  8  9
 *                       tanl  |  3  7 10 11 12    covar(i)
 *                        psi  |  4  8 11 13 14
 *                       q/pt  |  5  9 12 14 15
 *                       -----------------------
 *
 *                       and where phi  = atan2(y0,x0)
 *                                 R    = sqrt(x0*x0 + y0*y0)
 *                                 q/pt = icharge*invpt
 *
 *       chisq[2]        chi-square per degree of freedom where the
 *                       deg of freedom = (# pts used - # track params fit).
 *                       This may be the chi-square for the x-y circle fit
 *                       and the path-z linear fit.  Or this may be a
 *                       probability of chi-square. 
 */

 struct dst_track  {
  short      id;               /* Primary key                            */
  short      iflag;            /* bitmask quality information            */
  short      det_id;           /* Detector id information                */
  short      n_point;          /* number of points assigned to track,    */
                               /* SVT, TPC, FTPC component #s are packed */
  short      n_max_point;      /* maximum number of points possible,     */
                               /* SVT, TPC, FTPC component #s are packed */
  short      n_fit_point;      /* number of points used in fit,          */
                               /* SVT, TPC, FTPC component #s are packed */
  short      icharge;          /* Particle charge in units of |e|        */
  short      pid;              /* Geant particle ID for assumed mass     */
  long       id_start_vertex;  /* F.key to dst_vertex for starting vrtx. */
  float      x0;               /* x-coord. at start (cm)                 */
  float      y0;               /* y-coord. at start (cm)                 */
  float      z0;               /* z-coord. at start (cm)                 */
  float      psi;              /* azimuthal angle of pT vector (deg)     */
  float      tanl;             /* tan(dip) =pz/pt at start               */
  float      invpt;            /* 1/pt at start (GeV/c)^(-1)             */
  float      covar[15];        /* full covariance matrix                 */
  float      chisq[2];         /* Chi-square per degree of freedom       */
  float      x_first[3];       /* coord. of first measured point (cm)    */
  float      x_last[3];        /* coord. of last measured point (cm)     */
  float      length;           /* trk length from first to last pnt (cm) */
  float      impact;           /* impact parameter from prim. vertex (cm)*/
 } ;
/*  Last mod. for dst_track:   $Date: 1999/09/27 16:54:15 $ */

\end{verbatim}

\vspace{0.1in}
\leftline{\bf dst\_dedx:}
\vspace{0.05in}

The only changes here were to reduce the word sizes from `long' to `short.'

\begin{verbatim}

/*  dst_dedx.idl                   */
/*  Table:  dst_dedx               */

/*  This table contains event reconstruction information about the
 *  dE/dx for charged particle trajectories (tracks) listed in the
 *  dst_track table, first instance.  Contains dE/dx and number of used
 *  hits for SVT, TPC and FTPC.  It is filled with information for one
 *  event with one track detector segment entry per table row.  For tracks
 *  with both SVT and TPC dE/dx information this table fills two rows
 *  with the same foreign key (id_track) to the corresponding track entry
 *  in dst_track.
 */

 struct dst_dedx   {
  short  id_track;   /* Foreign key to dst_track                         */
  short  det_id;     /* Det ID-SVT,TPC,FTPC, use StDetectorDefinitions.h */
  short  iflag;      /* Quality/status of extacted dE/dx (future use)    */
  short  ndedx;      /* number of points used in dE/dx calcu.            */
  float  dedx[2];    /* Mean, most prob.; or other dE/dx and variance    */
 } ;
/*  Last mod. for dst_dedx:   $Date: 1999/09/27 16:54:15 $ */

\end{verbatim}


\vspace{0.1in}
\leftline{\bf dst\_vertex:}
\vspace{0.05in}

\begin{itemize}
\item
The vertex ID and number-of-daughters were reduced to `short.'
\item
The detector ID remains a `long' to accomodate Helen and Spiros' packing
algorithm.
\item
The vertex position error is replaced with the covariance matrix.
\item
The chi-square and probability of chi-square are both included.
\end{itemize}

\begin{verbatim}

/*  dst_vertex.idl           */
/*   Table: dst_vertex       */

/*  This table contains event reconstruction information about the
 *  various candidate vertices, including the primary vertex, V0,
 *  cascade, kinks and others.  It is filled with information for one
 *  event with one vertex candidate per table row. 
 *
 *  Description of selected variables:
 *  ----------------------------------
 *       n_daughters     For the primary vertex this counts only tracks
 *                       with SVT and/or TPC segments that were found to
 *                       have a DCA point within a 3D cut distance from
 *                       the primary vertex; the FTPC tracks are not counted. 
 *                         
 *       det_id          Indicates which detector(s) contributed space
 *                       points to the daughter tracks using a packing
 *                       algorithm and the detector ID definitions in
 *                       StDetectorDefinitions.h
 *
 *       covar[6]        Vertex fitting covariance matrix where:
 *                       ______|___________
 *                          x  |  1  2  4     
 *                          y  |  2  3  5       covar(i)
 *                          z  |  4  5  6    
 */

 struct dst_vertex  {
  short     vtx_id;        /* vertex type ID, use StDetectorDefinitions.h  */
  short     n_daughters;   /* number of charged daughter tracks            */
  long      id;            /* Primary key                                  */
  long      iflag;         /* bitmask of e.g. quality information, content */
                           /* varies depending on vertex type              */
  long      det_id;        /* Packed daughter track info (see above)       */
  long      id_aux_ent;    /* Foreign key into auxiliary vertex table(s)   */
  float     x;             /* vertex coordinates in                        */
  float     y;             /*   STAR global                                */
  float     z;             /*   coordinate system in (cm)                  */
  float     covar[6];      /* vertex position error covariance matrix      */
  float     chisq[2];      /* Chi-square per degree of freedom and probab. */
                           /* of chisq P(chi^2,ndf) of vertex fit          */
 } ;
/*  Last mod. for dst_vertex:   $Date: 1999/09/27 16:54:15 $ */

\end{verbatim}

\vspace{0.1in}
\leftline{\bf Remaining DST Tables:}
\vspace{0.05in}

All other DST tables not included here were not touched.  These include:
\begin{itemize}
\item
{\tt dst\_L0\_Trigger}
\item
{\tt dst\_L1\_Trigger}
\item
{\tt dst\_L2\_Trigger}
\item
{\tt dst\_TriggerDetectors}
\item
{\tt dst\_rch}
\item
{\tt dst\_tkf\_vertex}
\item
{\tt dst\_v0\_vertex}
\item
{\tt dst\_xi\_vertex}
\item
{\tt dst\_tof\_evt}
\item
{\tt dst\_tof\_trk}
\end{itemize}

Finally, the {\tt dst\_track\_aux} table can be deleted in any new software
releases.  The full track covariance matrix is now stored in the track
table; the residual averages are not necessary to save.  Also the
{\tt dst\_monitor\_hard} table can be deleted.

Please send comments to Lanny Ray, ray@physics.utexas.edu.

\end{document}
